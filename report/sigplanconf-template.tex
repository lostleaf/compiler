%-----------------------------------------------------------------------------
%
%               Template for sigplanconf LaTeX Class
%
% Name:         sigplanconf-template.tex
%
% Purpose:      A template for sigplanconf.cls, which is a LaTeX 2e class
%               file for SIGPLAN conference proceedings.
%
% Guide:        Refer to "Author's Guide to the ACM SIGPLAN Class,"
%               sigplanconf-guide.pdf
%
% Author:       Paul C. Anagnostopoulos
%               Windfall Software
%               978 371-2316
%               paul@windfall.com
%
% Created:      15 February 2005
%
%-----------------------------------------------------------------------------


\documentclass[nocopyrightspace]{sigplanconf}

% The following \documentclass options may be useful:
%
% 10pt          To set in 10-point type instead of 9-point.
% 11pt          To set in 11-point type instead of 9-point.
% authoryear    To obtain author/year citation style instead of numeric.

\usepackage{amsmath}

\begin{document}

\conferenceinfo{WXYZ '05}{date, City.} 
\copyrightyear{2005} 
\copyrightdata{[to be supplied]} 

%\titlebanner{banner above paper title}        % These are ignored unless
%\preprintfooter{short description of paper}   % 'preprint' option specified.

\title{Report for Compiler Course Project}
%\subtitle{Subtitle Text, if any}

\authorinfo{Qinglin Li}
           {Shanghai Jiaotong University}
           {lostleaf@icloud.com}

\maketitle

\begin{abstract}
This report describes the compiler course project. The design of Abstract syntax tree and immediate representative along with some optimization are included in this report. 
\end{abstract}

\category{D.3.4}{Programming Languages}{compiler}


\keywords
Compiler, Abstract Syntax Tree, Immediate Representative, Code Optimization, Register Allocation, Constant Folding

\section{Introduction}
This project aims at building a compiler for a subset of C language. It removed float numbers, some confusing grammars and most library functions in C language. And, of course, the compiler translate C code to MIPS code with ANTLR4 parser generate tool.

\section{Abstract Syntax Tree}
The Abstract Syntax Tree(AST for short) is generated while parsing and the whole process is contained in \textit{C.g4} file under \textit{parser} directory. \\
The AST is similar with Parsing Tree, but removes useless information. Every node in Parsing Tree is corresponded to a node in AST.\\
And since addition, multiplication, and other binary operator expressions are similar, a generic type is used here. And it brings much benefit when generate Immediate Representative.\\
The inheritance of AST is shown below:\\
\begin{itemize}
\item Node
\begin{itemize}
	\item Program
	\item Declaration
	\item $\cdots$
	\item Stmt(Correspond to Statement)
	\begin{itemize}
		\item CompStmt(Correspond to Compound-Statement)
		\item $\cdots$
	\end{itemize}
	\item Expression
	\begin{itemize}
		\item AssExpr(Correspond to Assignment-Statement)
		\item BinExpr$<\text{ExprType}>$(generic type)
		\begin{itemize}
			\item AddExpr:BinExpr$<\text{MulExpr}>$
			\item MulExpr:BinExpr$<\text{CastExpr}>$
			\item $\cdots$
		\end{itemize}
	\end{itemize}
\end{itemize}
\end{itemize}


\section{Semantic Checking}
The semantic checking producure is called after AST generating. Semantic checking mainly check the following items:
\begin{enumerate}
\item Type
\item Left value
\item Declaration and use before declaraed
\item Other items including breaks, returns, etc.
\end{enumerate}

\subsection{Type}
	Types all have upcase class names in case of mixing up with Java type names. \\
	The inheritance:
	\begin{itemize}
		\item TYPE
		\begin{itemize}
			\item CHAR
			\item FUNCTION
			\item INT
			\item VOID
			\item STRING
			\item NAME
			\item POINTER
			\begin{itemize}
				\item ARRAY
			\end{itemize}
			\item RECORD
			\begin{itemize}
				\item UNION
				\item STRUCT
			\end{itemize}
		\end{itemize}
	\end{itemize}
	\textbf{CHAR},\textbf{ INT}, and \textbf{VOID} are singleton classes.\\
	Type checking mainly happened in expressions and some statements.\\
	In expressions, if the oprands and operators doesn't match, a error would be reported. For instance, if a structure is multiplied by an integer, a ``type not match'' error would be reported.\\
	And some statements require special types. For instance, the condition of \textbf{if} or \textbf{while} statements must be integer.

\subsection{Left Value}
	Most left values checking happens in assignments. And some operator such as ``\&''(get address) and ``++''(self increment)\\
	
\subsection{Declaration}
	The check about declaration and use before declarated is based on symbol table. If a variable cannot be found in the symbol table while used, a ``variable not declaraded'' error would be reported.
	
\subsection{Returns and Breaks}
	Returns and breaks are checked by some counters.

\subsection{Other items}
	The details not metioned can be found in \textit{Semantic.java} under \textit{semantic} directory

\appendix
\section{Appendix Title}

This is the text of the appendix, if you need one.

\acks

Acknowledgments, if needed.

% We recommend abbrvnat bibliography style.

\bibliographystyle{abbrvnat}

% The bibliography should be embedded for final submission.

\begin{thebibliography}{}
\softraggedright

\bibitem[Smith et~al.(2009)Smith, Jones]{smith02}
P. Q. Smith, and X. Y. Jones. ...reference text...

\end{thebibliography}

\end{document}
